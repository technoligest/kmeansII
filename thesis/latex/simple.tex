% simple.tex -- a very simple thesis document for demonstrating
%   dalthesis.cls class file
\documentclass[12pt]{dalthesis}

\begin{document}

\title{Kmeans Clustering}
\author{Yaser Alkayale}

% The following degrees are included in the current dalthesis.cls
% class file:
\mcs  % options are \mcs, \macs, \mec, \mhi, \phd, and \bcshon

% If you degree is not included, you can set several options manually.
% The following example shows the parameters for the \mcs degree.
% However, if you need to set these parameters manually, please check
% the correct names with the Faculty of Graduate Studies, and let the
% maintainer of this class file know (Vlado Keselj, vlado@cs.dal.ca).
% MCS Example:

\degree{Bachelor's of Computer Science, Honours}
\degreeinitial{B.C.Sc.}
\faculty{Computer Science}
\dept{Faculty of Computer Science}

% Month and Year of Defence
\defencemonth{April}\defenceyear{2018}

\dedicate{This thesis is in dedication to my grandfather whom I was named after.}

% This sample thesis contains no tables nor figures, so there is no
% need to include lists of tables and figures in the front matter:
\nolistoftables
\nolistoffigures

\frontmatter

\begin{abstract}
  Clustering is a well-known task that has been studied and used for decades. The idea is to take a set of items and group them into a number of clusters based on a similarity measure. K-means proposed in 1957 by Stuart Lloyd is one of the most widely used clustering algorithm and is still used today for its reasonably fast heuristic to find the clusters based on the Lloyd algorithm and more recent developments in that area. K-means has two main parts to clustering, the initial seeding process and the iteration process. The seeding process picks k initial seeds as cluster centres, and highly affects the accuracy of the final result in the algorithm. The iteration process dominates running time to move the centres around until it converges to an optimum. In this paper, we discuss a new method of the seeding process that gives us more accurate seeds to start the algorithm. We also discuss a novel approach to find an approximation of the correct number of clusters for a given dataset.
\end{abstract}

\begin{acknowledgements}
Thanks to all the little people who make me look tall.
\end{acknowledgements}

\mainmatter

\chapter{Introduction}

Get it done!  Use reference material by Limpet~\cite{latex-by-lamport} or
Gooses, Mittelback, and Samarin~\cite{latex-companion}.

To test if the margins are satisfactory, let us generate a lot of
garbage text:
This sentence goes on, and on, and on, and on,
and on, and on, and on, and on, and on, and on, and on, and on, and on,
and on, and on, and on, and on, and on, and on, and on, and on, and on,
and on, and on, and on, and on, and on, and on, and on, and on, and on,
and on, and on, and on, and on, and on, and on, and on, and on, and on,
and on, and on, and on, and on, and on, and on, and on, and on, and on,
and on, and on, and on, and on, and on, and on, and on, and on, and on,
and on, and on, and on, and on, and on, and on, and on, and on, and on,
and on, and on, and on, and on, and on, and on, and on, and on, and on,
and on, and on, and on, and on, and on, and on, and on, and on, and on,
and on, and on, and on, and on, and on, and on, and on, and on, and on,
and on, and on, and on, and on, and on, and on, and on, and on, and on,
and on, and on, and on, and on, and on, and on, and on, and on, and on,
and on, and on, and on, and on, and on, and on, and on, and on, and on,
and on, and on, and on, and on, and on, and on, and on, and on, and on,
and on, and on, and on, and on, and on, and on, and on, and on, and on,
and on, and on, and on, and on, and on, and on, and on, and on, and on,
and on, and on, and on, and on, and on, and on, and on, and on, and on,
and on, and on, and on, and on, and on, and on, and on, and on, and on,
and on, and on, and on, and on, and on, and on, and on, and on, and on,
and on, and on, and on, and on, and on, and on, and on, and on, and on,
and on, and on, and on, and on, and on, and on, and on, and on, and on,
and on, and on, and on, and on, and on, and on, and on, and on, and on,
and on, and on, and on, and on, and on, and on, and on, and on, and on,
and on, and on, and on, and on, and on, and on, and on, and on, and on,
and on, and on, and on, and on, and on, and on, and on, and on, and on,
and on, and
and here we should be around top of the page~2, and we go on, and on, and on, and on, and on.
This following line \rule{5cm}{1pt} should be exactly 5cm long.  It
can be used to check the typesetting process.
And now we go on,
and on, and on, and on, and on, and on, and on, and on, and on, and on,
and on, and on, and on, and on, and on, and on, and on, and on, and on,
and on, and on, and on, and on, and on, and on, and on, and on, and on,
and on, and on, and on, and on, and on, and on, and on, and on, and on,
and on, and on, and on, and on, and on, and on, and on, and on, and on,
and on, and on, and on, and on, and on, and on, and on, and on, and on,
and on, and on, and on, and on, and on, and on, and on, and on, and on,
and on, and on, and on, and on, and on, and on, and on, and on, and on,
and on, and on, and on, and on, and on, and on, and on, and on, and on,
and on, and on, and on, and on, and on, and on, and on, and on, and on,
and on, and on, and on, and on, and on, and on, and on, and on, and on,
and on, and on, and on, and on, and on, and on, and on, and on, and on,
and on, and on, and on, and on, and on, and on, and on, and on, and on,
and on, and on, and on, and on, and on, and on, and on, and on, and on,
and on, and on, and on, and on, and on, and on, and on, and on, and on,
and on, and on, and on, and on, and on, and on, and on, and on, and on,
and on, and on, and on, and on, and on, and on, and on, and on, and on,
and on, and on, and on, and on, and on, and on, and on, and on, and on,
and on, and on, and on, and on, and on, and on, and on, and on, and on,
and on, and on, and on, and on, and on, and on, and on, and on, and on,
and on, and on, and on, and on, and on, and on, and on, and on, and on,
and on, and on, and on, and on, and on, and on, and on, and on, and on,
and on, and on, and on, and on, and on, and on, and on, and on, and on,
and on, and on, and on, and on, and on, and on, and on, and on, and on,
and on, and on, and on, and on, and on, and on, and on, and on, and on,
and on, and on, and on, and on, and on, and on, and on, and on, and on,
and on, and on, and on, and on, and on, and on, and on, and on, and on,
and on, and on, and on, and on, and on, and on, and on, and on, and on,
and on, and on, and on, and on, and on, and on, and on, and on, and on,
and on, and on, and on, and on, and on, and on, and on, and on, and on,
and on, and on, and on, and on, and on, and on, and on, and on, and on,
and on, and on, and on, and on, and on, and on, and on, and on, and on,
and on, and on, and on, and on, and on, and on, and on, and on, and on,
and on, and on, and on, and on, and on, and on, and on, and on, and on,
and on, and on, and on, and on, and on, and on, and on, and on, and on,
and on, and on, and on, and on, and on, and on, and on, and on, and on,
and on, and on, and on, and on, and on, and on, and on, and on, and on,
and on, and on, and on, and on, and on, and on, and on, and on, and on,
and on, and on, and on, and on, and on, and on, and on, and on, and on,
and on, and on, and on, and on, and on, and on, and on, and on, and on,
and on, and on, and on, and on, and on, and on, and on, and on, and on,
and on.


\chapter{Doing It}

\section{Getting Ready}

Get all the parts that I need.  I can throw in a whole pile of terms like
preparation,
methodology,
forethought,
and
analysis
as examples for me to use in the future.

\section{Next Step}

Do it!

Of course, you have to have pictures to show how you did it to make people
understand things better.

\chapter{Conclusion}

Did it!\cite{latex-by-lamport}

\bibliographystyle{plain}
\bibliography{simple}

\end{document}

% You may ignore or delete these two lines of comments.
% $Id: simple.tex 386 2012-11-12 15:11:16Z vlado $
